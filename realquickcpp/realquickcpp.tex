\documentclass[11pt]{article}
\usepackage{graphicx}
\usepackage{amssymb}
\usepackage{tex2page}
\usepackage[noend]{algorithmic}
\usepackage{algorithm}
\usepackage{amsmath,amsthm}
\usepackage{xy}

\texonly
\textwidth = 6 in
\textheight = 9 in
\oddsidemargin = 0.0 in
\evensidemargin = 0.0 in
\topmargin = 0.0 in
\headheight = 0.0 in
\headsep = 0.0 in
\parskip = 0.1in
 \parindent = 0.0in
 \endtexonly
 
 \newcommand {\marginal}[1]{\mbox{}\marginpar{\raggedleft\hspace{0pt}#1}}
 
 \scmkeyword{asm auto bool break case catch char class const const_cast continue default delete do double dynamic_cast else enum explicit export extern false float for friend goto if inline int long mutable namespace new operator private protected public register reinterpret_cast return short signed sizeof static static_cast struct switch template this throw true try typedef typeid typename union unsigned using virtual void volatile wchar_t while map vector}

\cssblock
.navigation     {font-family: verdana, arial, sans serif; font-style: bold; font-size:11px; line-height:16px}
\endcssblock

%\htmlmathstyle{no-image}
\inputcss body.css

\title{Real Quick C++}
\author{Will Fitzgerald}
\begin{document}
\maketitle
\tableofcontents

\newpage

\section{Introduction}

A delightful and sprightly introduction.

%Broadly construed, my research interests  are in natural language
%understanding and models of task execution and monitoring. Typically,
%the approach I take to problems in these domains is to investigate how
%to integrate symbolic models with statistical, information
%theoretical, or dynamic data. For example, my dissertation work
%focused on describing and building natural language parsers that could
%be quickly incorporated into computer systems \marginal{Are they \textit{interests} or \textit{something else}?\\ See \urlh{http://www.help.com}{the Help.com site}.}such as intelligent
%tutors and expertise browsers. It described a class of parsing
%technologies (indexed concept parsing) which combined
%information-theoretic and knowledge-based approaches to natural
%language understanding. Several systems based on indexed concept
%parsing have been commercially developed and are in use for web-based
%knowledge management and for tutoring. 

%\begin{verbatim}
% (define fact 
%   (lambda (n)
%     (if (= n 0)
%         1 ; the base case
%         (* n (fact (- n 1))))))
%\end{verbatim}
%                   
%Since 1996, my colleague R. James Firby and I have been investigating
%and building systems that integrate natural language understanding and
%generation, task execution, and recognition of complex event
%patterns. In several NASA SIBIR (Small Business Innovation Research)
%projects, as well as part of a commercial R\&D effort, we have worked
%on representations and algorithms for conceptual memory storage and
%querying, semantically-oriented natural language understanding, and
%dialogue and task execution. This work has resulted in several
%commercial applications for automotive telematics, network and system
%monitoring, and is actively used at NASA's Johnson Space Center for
%several projects. 

%I have been interested in investigating the similarities and
%differences between event recognition and monitoring, and natural
%language understanding. I have created a language for events and event
%pattern recognition that incorporates ideas from  natural language
%parsing which takes into account the multi-channel and often
%non-serial nature of events, while incorporating the hierarchical and
%serial nature of language.  

%\scm{
%(define fact
%"The factorial function"
% (lambda (n)
%     (if (= n 0) 1 ;the base case
%        (* n (fact (- n 1))))))
%}

%These interests have led to ancillary interests in programming
%languages (both as a tool user and a tool design), operating systems
%and distributed computing, cognitive science, linguistics and
%philosophy of language. I am also interested in ``humane computing,''
%that is, the interaction between computational devices and approaches,
%and people and how this should inform design of computational devices.  

\newpage
\section{Beyond ``Hello World!''}

The typical first program to write in any language is ``Hello, World!,'' and you can find many examples of tutorials that do just that. \marginal{Fix: \urlh{http://www.google.com}{Search Google for C, hello, world}.} What we want to do is someone real, as well as quick. Our first program will read text from the standard input port, collect statistics on the frequencies of letters in that text, and print the frequencies on the standard output port.

Here is the program to do this:

\scm{
#include <map>
#include <iostream>
using namespace std;

int main() 
 {
  map<char, int> freqs; 
  char ch;
  
  while (cin .get(ch))
    freqs[ch]++;
 
  for (map<char,int>::iterator it = freqs.begin(); it != freqs.end(); it++) 
  {
    cout << it->first << "\t" << it->second << "\n";
  }
}
}

And here is the output from running the resulting program on this program code:

\verb{
	16
 	55
!	1
"	4
#	2
(	6
)	6
+	4
,	2
-	2
.	3
/	2
:	2
;	8
<	14
=	2
>	6
[	1
\	2
]	1
a	11
b	1
c	13
d	5
e	16
f	7
g	3
h	7
i	18
l	3
m	6
n	13
o	6
p	4
q	5
r	13
s	11
t	17
u	5
w	1
{	2
}	2
}

Isn't that grand?

%Oddly, this has led me to write code that looks like:
%\scm{
%#include <iostream> 
%using namespace std;
%// an 'enum' is a list of values
%;; Wish I could change the quoting ... but an enum is still a list of values
%enum SignalValueType 
% {SignalCharacter, SignalInteger, SignalDouble, SignalLong, SignalBoolean}; 

%class Signal {
%public:
%  Signal() {
%    theType = SignalInteger;
%    ti = 0;
%  }
%  Signal(int i) {
%    theType = SignalInteger;
%    ti = i;
%  }
%  Signal(char ch) {
%    theType = SignalCharacter;
%    tch = ch;
%  }
%  Signal(double d) {
%    theType = SignalDouble;
%    td = d;
%  }
%  Signal(long l) {
%    theType = SignalLong;
%    tl = l;
%  }
%  Signal(bool b) {
%    theType = SignalBoolean;
%    tb = b;
%  }
%  bool equals(const Signal& probe) {
%    switch (theType) {
%    case SignalInteger: 
%      return (probe.theType==SignalInteger) && (probe.ti==ti);
%    case SignalCharacter: 
%      return (probe.theType==SignalCharacter) && (probe.tch==tch);
%    case SignalDouble: 
%      return (probe.theType==SignalDouble) && (probe.td==td);
%    case SignalLong: 
%      return (probe.theType==SignalLong) && (probe.tl==tl);
%    case SignalBoolean: 
%      return (probe.theType==SignalBoolean) && (probe.tb==tb);
%    default: return 0;
%    };
%  };

%private:
%  SignalValueType theType;
%  union {
%    char tch;
%    int ti;
%    double td;
%    long tl;
%    bool tb;
%  };
%};

%int main()
%{
%  Signal s = Signal();
%  int ii= 0;
%  Signal s1 = Signal(ii);
%  Signal bs = Signal(true);
%  cout << "Self similar: " << s1.equals(s1) << "\n";
%  cout << "Same value: " << s.equals(s1) << "\n";
%  cout << "Int is bool " << s1.equals(bs) << "\n";
%} 
% 
%}

\end{document}