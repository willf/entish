\documentclass[11pt]{article}

%\usepackage{epsfig}
%\usepackage{graphicx}
\usepackage{makeidx}
\usepackage{times}
\usepackage{amsmath}
\usepackage{tex2page}


\begin{document}
\title{Only Hope}
\externaltitle{Only Hope}
\author{Will Fitzgerald}

\inputcss body.css
\maketitle

I'd like to thank Sandy for being the liturgist today, and John
Coates, of course, for playing. It is on Sundays such as today
that we realize how much we miss having Linda with us. It's an
honor for me to be able to preach today, because I'm not a
preacher of course. You may be aware that in the fall I'll be a
teacher again-teaching computer science at Kalamazoo College, and
so perhaps I should teach today, but what I really want to do is
preach-to work though these great words of Paul's and of Jesus's,
and to maybe speak a word of encouragement or hope.

What I have been doing for the past eight years or so is to write
computer software, along with some others. It's work I've really
enjoyed doing, because we're trying to do something hard and maybe
a little bit odd-to get computers to understand things like we do
and understand language the way we talk, instead of the usual
other way round. And we've made some progress, but I want to tell
you a story about something that happened a few years ago.

When you're part of a group of people starting a business, you
have to make a lot of decisions together, from relatively minor
(but contentious) things such as what the name of the company
should be and what the logo should look like, to what the product
you're making is, and who it is you're going to sell it to.

One policy that I brought forward and was really important to me,
and that I convinced our team of was this: we shouldn't actively
market or develop software for the military or military
contractors. This came out of my own growing convictions that it
was important for me to follow in the way of peace-Jesus calls us
to be peacemakers, and developing for the war makers didn't seem
consistent with that call. On the one hand, this didn't have a big
impact on us. We were trying to write software to help salespeople
do a better of selling-you can decide whether that was a good use
of our time and abilities or not.

It turns out that we were not really very successful in convincing
salespeople to buy our software (maybe we should have used our own
software to help us do the selling), and one day, I got an email
from someone I know at Lockheed-Martin. Lockheed-Martin and Boeing
were, at the time, competing to win a 19 billion dollar contract
to build the Joint Strike Fighter, which is to be used by the US
Navy, Air Force and Marines, as well as the military in Great
Britain and potentially other countries. This person did
simulations. Basically, what he wanted to do was to use our
software to write little software versions of the Joint Strike
Fighter and little software versions of, say, North Korean tanks
and have the little software Joint Strike Fighters fly around and
drop little software bombs on the software tanks and kill the
little software North Korean soldiers inside. Of course, the point
of this was to help the military build big, real fighters that
drop real, big bombs on enemy tanks and people and other targets.

As I said, I didn't think that developing software like this was
consistent with my personal beliefs, and our executive team had
decided not to work on  software contracts for military
applications. So, of course you know what I told the person at
Lockheed-Martin who wanted our software and our expertise to help
them win their bid to build the next generation of fighters.

I said, "Yes, we'd love to."

To this day, I don't know why I said yes. Maybe it was because I
don't like to say no. Maybe I was a little desperate to have some
money coming into the company at a time when there wasn't much
coming in. But I went against my principles and the policy of our
company and said yes. The consequences of this were strange. I
lost a little bit of respect for myself, and I think respect from
others. And it left me in a moral quandary about whether I should
continue working on the contract we signed.  Well, in the end, the
company I was working for went out of business and Lockheed-Martin
won the Joint Strike Fighter contract. It's of little comfort
knowing that our software probably did not play a significant role
in this.

When Paul says,  "I do not understand my own actions. For I do not
do what I want, but I do the very thing I hate." and "I can will
what is right, but I cannot do it. For I do not do the good I
want, but the evil I do not want is what I do," I don't see Paul
making some point of theology, but describing the struggle that
he, and you, and I face day to day.

As a point of theology, we might disagree about the depths of our
inability to do what is right and keep from doing what is wrong.
My copy of the Heidelberg Catechism says we are "poisoned," and I
think this is a good image-I feel a little bit sick thinking about
what I did then.

In this passage, Paul makes a contrast between "flesh" and
"spirit." I think that the word in Greek for "flesh," is
particularly ugly: "sarx." In other places, Paul lists specifics:
sexual immorality, impurity and debauchery;  idolatry and
witchcraft; hatred, discord, jealousy, fits of rage, selfish
ambition, dissensions, factions  and envy; drunkenness, orgies,
and the like (from Galatians 5). It's clear that not all of these
are literally sins of the flesh: my sin of selfish ambition, for
example. But it's just as clear from Paul and from our own
experience that our flesh is a battleground, whether that's sexual
temptation, or drug or alcohol addiction.

In any case, the struggle is real, and it's a struggle we all
face, a burden that we all bear. In fact, says, Paul, it's a
struggle we often lose, no matter how hard we try, or how much
help we have from others, or what our best intentions are. And I
think if we are honest with ourselves, we know this to be true. No
wonder Paul screams out that he feels wretched!

"Who will rescue me from this body of death?," asks Paul. And he
answers his own question: "Thanks be to God through Jesus Christ
our Lord!" Paul points to Jesus as the way out.

I'm grateful to whomever put together the lectionary that the
particular passage from Matthew is used. We have Paul, describing
the heavy burden of the struggle each one of us faces, pointing to
Jesus as the one who will rescue. And now we have Jesus saying,
"Come to me, all you that are weary and are carrying heavy
burdens, and I will give you rest. Take my yoke upon you, and
learn from me; for I am gentle and humble in heart, and you will
find rest for your souls. For my yoke is easy, and my burden is
light."

On the face of it, a yoke doesn't seem like something that is
going to ease our burden. A yoke, of course, is what you put on a
team of oxen or other animals to attach them to a plow or cart or
some other burden. What does Jesus mean?

One thing I think it is fair to say is this: Jesus acknowledges
that our burdens will not go away. He's offering help, but he's
not denying what we all know to be true: we will have struggles,
we will have burdens, we will have hardships. I think it's true
that he offers, as it were, a trade-up from some of our burdens to
some of his. We can, for example, carry a burden for those who are
poor and at the margins instead of the burden of, say, an anxious
concern to look good and have all the right things. Still, I don't
think songs with words like, "And now I am happy all the day!"
describe what our life as followers of Jesus is really like. On
Friday, some of us went to the funeral of Eula Bailey, North
Presbyterian Church's oldest member, who died just six weeks short
of turning 100 years old. Our pastor emeritus, Bob Rasmussen,
spoke at the graveside service, from the 23rd psalm, which reminds
us, though we "fear no evil," we still walk through the "valley of
the shadow of death." And, so, although Jesus offers us "rest,"
he's not offering us a magic drug that makes every pain and burden
go away. The yoke may be "easy" but the burden remains.

In a few weeks, most of our family is going on a canoe trip to the
Minnesota/Canada Boundary Waters Wilderness Area, and, frankly,
I'm a little worried. You can tell by looking at me that I'm not
your typical wilderness or sports enthusiast. Because it's a
wilderness area, we're going to have to bring everything we want
to have with us-our shelter, our food, our clothing and gear, and
our canoes. And we'll be going on portages between lakes, which
means we'll have to pick everything up and carry our stuff
overland. And I'm hoping that when I pick up the canoe or pack, it
won't be too heavy. And, just as important, I'm hoping that the
straps on the pack will be well fitted to my back, or (if I'm
carrying a canoe) that I'll get just the right fulcrum point so
the canoe will be as easy as possible to manage. And I think this
is an example of another thing Jesus is saying: the burdens we
have to carry won't be greater than we can bear, and we'll be able
to get the right grip on them.

There's another aspect of the "easy yoke" I want to talk about,
and I'm sure you've heard this before, but it's too important not
to mention. Yokes are typically built, not for one alone, but for
two. When Jesus says, "take my yoke upon you," he's promising to
be beside us, pulling the burden with us. I think we could spend
the rest of today telling stories about how we've found this to be
true-in certain difficult times, we've had the strong sense that
Jesus is with us, helping us. Can I get an amen?

Now, I'm not saying that we'll always feel this. And I have to
admit that this hasn't often been the case for me. But, just as I
know and am certain that Linda MacDonald, our pastor, is thinking
about us this morning as she spends what is her afternoon in
Germany, even though we can't see her or speak to her, I am even
more certain that Jesus is with us this morning, bearing the yoke
of our struggles and burdens.

My daughter Jane and I have been doing a little catechism class
together, roughly based on the Heidelberg Catechism, which is one
of the confessions, or statements of belief, of our church.  The
first question and the first answer are:

Q1. What is your only hope in life and in death? A1. That I
belong, body and soul, in life and in death, not to myself, but to
my faithful savior, Jesus Christ.

I take this to mean that Jesus Christ is our best and last hope.

I take this to mean that Jesus Christ is our only hope even when
we feel most alive, most in control of ourselves and ready to take
on the world.

I take this to mean that Jesus Christ is our only hope when we
pass through the valley of the shadow of death, and remember that
death awaits us all.

I take this to mean that Jesus Christ is our only hope when our
body is well and when our body is sick.

I take this to mean that Jesus Christ is our only hope when I feel
forsaken and unconnected.

I take this to mean that Jesus Christ is our only hope when my
self-esteem is high and when my self-esteem is low.

I take this to mean that Jesus Christ is our only hope when I am
faithless and full of selfish ambition.

I take this to mean that Jesus Christ is our only hope when I
struggle to do what is right and fail.

I take this to mean that Jesus Christ is our only hope when I
struggle to avoid what is wrong and fail.

I take this to mean that our only hope is to belong, body and
soul, in life and in death, not to ourselves, but to our faithful
savior, Jesus Christ.

What is your only hope in life and in death?

\end{document}
